% Nu ik op het einde ben van mijn bacheloropleiding aan de Hogeschool PXL ben ik vrij zeker dat ik ga schakelen naar een masteropleiding. Dit gaat zeker een uitdaging worden, maar in mijn huidige opleiding heb ik geleerd dat zeer veel mogelijk is als ik mij er volledig op gooi.

% Tijdens mijn opleiding is mij duidelijk geworden dat informatica niet enkel programmeren is maar er veel meer bij komt kijken. De sociale aspecten, die volgens de stereotypes er niet is, is enorm belangrijk en deze heb ik kunnen bemachtigen door verschillende vakken te volgen en opdrachten uit te voeren tijdens de academiejaren.

% Door de vele groepswerken en zaken als de buitenlandse reis, ben ik veel in contact ge komen met mijn medestudenten. Hierdoor leerde ik hoe ik niet alleen professioneel met ze om moest gaan, maar ook als vrienden. Ook bijvoorbeeld de Care-athon heeft mijn mentaliteit doen wijzigen, het helpen van mensen die het effectief nodig hebben is in mijn ogen zeer belonend en brengt mij veel voldoening.

% De afgelopen drie jaar ben ik ook heel wat veranderd in de zin van durven. In de plaats van aan zeer doordacht projecten opstellen rekening houdend met de zaken die ik al kon, stel ik nu een project op naar het resultaat dat ik wil bekomen. Dan zoek ik uit welke kennis en vaardigheden ik hier voor nodig heb en leer ik ze mezelf aan indien nodig. Zo begin ik aan projecten die veel uitdagender en leerrijker zijn dan ik een paar jaar geleden niet zou durven.

% De studiereis naar Berlijn heeft mij cultureel wel mijn ogen open gedaan. Ik had weinig besef van hoe mensen in de oorlog geleefd hebben en wat voor vreselijke zaken er gebeurd zijn. Maar ook de heropbouw en hoe een land hier van terug kan komen was indrukwekkend. Het heeft mij anders doen kijken naar de zaken rondom mij en mijn problemen doen relativeren.

% Het IT\hyp{}project heeft mij leren samenwerken met andere specialisaties zoals een grafisch ontwerper, een projectleider, een hulpverlener enzovoort. Met al deze mensen hebben we een project kunnen maken dat toch één geheel vormt. Ik heb leren mijn kennis overbrengen naar hun en leren luisteren naar hun specialiteiten.

% Ik heb ook leren uit mijn comfortzone te stappen door risico's en uitdagingen aan te gaan. Zo besloot ik om een splinter nieuwe keuzetraject te volgen, genaamd AI \& Robotics. Hier hadden wij als klas veel inspraak omdat de lessen nog zeer nieuw waren, onze feedback was essentieel. Het was niet de meest voor de hand liggende keuze om dit keuzetraject op te nemen, maar ik ben hier zeer tevreden over. Ik heb passies gevonden door mijn grenzen te verleggen en er volledig voor te gaan.

% Ik zou graag iedereen aan de Hogeschool PXL willen bedanken die mij geholpen heeft doorheen mijn opleiding, de mensen die mij kennis hebben bijgebracht en zij die mij uitgedaagd hebben. Zonder hen zou ik niet staan waar ik nu zou staan en daar ben ik ze dankbaar voor.

Nu ik op het einde ben van mijn bacheloropleiding aan de Hogeschool PXL ben ik vrij zeker dat ik ga schakelen naar een masteropleiding. Dit gaat zeker een uitdaging worden, maar in mijn huidige opleiding heb ik geleerd dat zeer veel mogelijk is als ik mij er volledig op gooi.

Tijdens mijn opleiding is mij duidelijk geworden dat informatica niet enkel programmeren is maar er veel meer bij komt kijken. De sociale aspecten, die volgens de stereotypes er niet is, is enorm belangrijk en deze heb ik kunnen bemachtigen door verschillende vakken te volgen en opdrachten uit te voeren tijdens de academiejaren.

Door de vele groepswerken en zaken als de buitenlandse reis, ben ik veel in contact ge komen met mijn medestudenten. Hierdoor leerde ik hoe ik niet alleen professioneel met ze om moest gaan, maar ook als vrienden. Ook bijvoorbeeld de Care\hyp{}athon heeft mijn mentaliteit doen wijzigen, het helpen van mensen die het effectief nodig hebben is in mijn ogen zeer belonend en brengt mij veel voldoening.

De afgelopen drie jaar ben ik ook heel wat veranderd in de zin van durven. In de plaats van aan zeer doordacht projecten opstellen rekening houdend met de zaken die ik al kon, stel ik nu een project op naar het resultaat dat ik wil bekomen. Dan zoek ik uit welke kennis en vaardigheden ik hier voor nodig heb en leer ik ze mezelf aan indien nodig. Zo begin ik aan projecten die veel uitdagender en leerrijker zijn dan ik een paar jaar geleden niet zou durven.

De studiereis naar Berlijn heeft mij cultureel wel mijn ogen open gedaan. Ik had weinig besef van hoe mensen in de oorlog geleefd hebben en wat voor vreselijke zaken er gebeurd zijn. Maar ook de heropbouw en hoe een land hier van terug kan komen was indrukwekkend. Het heeft mij anders doen kijken naar de zaken rondom mij en mijn problemen doen relativeren.

Het IT\hyp{}project heeft mij leren samenwerken met andere specialisaties zoals een grafisch ontwerper, een projectleider, een hulpverlener enzovoort. Met al deze mensen hebben we een project kunnen maken dat toch één geheel vormt. Ik heb leren mijn kennis overbrengen naar hun en leren luisteren naar hun specialiteiten.

Ik heb ook leren uit mijn comfortzone te stappen door risico’s en uitdagingen aan te gaan. Zo besloot ik om een splinter nieuwe keuzetraject te volgen, genaamd AI \& Robotics. Hier hadden wij als klas veel inspraak omdat de lessen nog zeer nieuw waren, onze feedback was essentieel. Het was niet de meest voor de hand liggende keuze om dit keuzetraject op te nemen, maar ik ben hier zeer tevreden over. Ik heb passies gevonden door mijn grenzen te verleggen en er volledig voor te gaan.

Ik zou graag iedereen aan de Hogeschool PXL willen bedanken die mij geholpen heeft doorheen mijn opleiding, de mensen die mij kennis hebben bijgebracht en zij die mij uitgedaagd hebben. Zonder hen zou ik niet staan waar ik nu zou staan en daar ben ik ze dankbaar voor.