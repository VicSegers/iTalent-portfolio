\subsubsection{Sinterklaas programming challenge}

\begin{tabular}{l l}
  Locatie: & n.v.t.\\
  Datum: & 03/12/2018 - 04/12/2018\\
  Duur: & 2 uur
\end{tabular}

Tijdens het vak Java Advanced, kregen we de mogelijkheid om aan een programming challenge deel te nemen. Er moest aan de hand van een gegeven technologie een probleem opgelost worden voor de sint. Ik heb niet enkel deelgenomen aan deze challenge, maar ik heb voor mijn resultaat ook een prijs gewonnen.

\subsubsection{\LaTeX{} - Softwaresysteem voor het zetten van documenten}

\begin{tabular}{l l}
  Locatie: & n.v.t.\\
  Datum: & 24/02/2020 - 20/04/2020\\
  Duur: & 20 uur
\end{tabular}

Na veel opzoekwerk over \LaTeX{}, kreeg ik reeds zin om het mezelf aan te leren en ermee te werken. Dit softwaresysteem gebruikte ik niet enkel om mijn eindwerk, maar ook mijn portfolio voor iTalent te schrijven. Ik legde mezelf op om deze techniek in voldoende mate onder de knie te krijgen om ze te kunnen toepassen wanneer dit nodig bleek.

\subsubsection{Website portfolio iTalent}

\begin{tabular}{l l}
  Locatie: & n.v.t.\\
  Datum: & 12/04/2020 - 20/04/2020\\
  Duur: & 12 uur
\end{tabular}

Voor het vak iTalent schreef ik mijn portfolio, om zo een overzicht te krijgen van mijn ervaringen die ik heb opgedaan gedurende de afgelopen 3 jaar. Omdat ik dit portfolio ook online beschikbaar wilde hebben en om dit ook mooier te kunnen presenteren dan een tekstdocument, heb ik ook een website geschreven. Zo leerde ik een statische responsieve website te maken met Bootstrap en het te hosten met Github Pages. Deze technologie"en had ik voordien nog niet gebruikt.