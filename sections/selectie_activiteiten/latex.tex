\LaTeX{} is een softwaresysteem dat je kan vergelijken met een tekstverwerker zoals Microsoft Word. Het is zeer populair in de wetenschappelijke wereld, want het blinkt uit in het maken van technische documenten. Het idee van \LaTeX{} is WYMIWYG, "What you mean is what you get", en dus een markup\hyp{}taal.

Ik kwam als eerst in contact met \LaTeX{} door de tekstverwerker van MacOS genaamd Pages. Hier kun je wiskundige vergelijkingen invoegen via \LaTeX{} en MathML. Eerst kopi\"eerde ik formules die ik online vond of ik converteerde ze via een online converter. Daarna zag ik dat de taal niet zo moeilijk te begrijpen was en zo kon ik mijn eigen formules uitschrijven.

Later kwam ik er achter dat \LaTeX{} niet enkel gemaakt is voor wiskundige formules weer te geven, maar ook kan gebruikt worden als een volwaardige tekstverwerker. Omdat het zeer veel gebruikt wordt om papers mee te schrijven, had ik besloten om mijn eindwerk hiermee te schrijven.

