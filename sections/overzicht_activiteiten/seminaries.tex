\subsubsection{AariXa - Docker for Dev and Ops}
% https://pxl-digital.pxl.be/page/seminarie-aarixa-27-02

\begin{tabular}{l l}
  Locatie: & PXL, B141\\
  Datum: & 27/02/2019\\
  Duur: & 3 uur
\end{tabular}

Het seminarie werd gegeven door Sven Luts, een software engineer bij AariXa. De spreker gaf een introductie over Docker en zijn onderliggende werking. Door middel van demo's werden de nodige commando's aan ons uitgelegd en gedemonstreerd.

\subsubsection{Belfius - Introduction to Artificial Intelligence Through Practice}
% https://pxl-digital.pxl.be/page/seminarie-belfius-20-03

\begin{tabular}{l l}
  Locatie: & PXL, B151\\
  Datum: & 20/03/2019\\
  Duur: & 3 uur
\end{tabular}

We kregen een introductie tot artifici"ele intelligentie door Jerome Fortias, AI practice leader bij BrightKnight. Er kwamen verschillende onderwerpen aan bod zoals de geschiedenis van artifici"ele intelligentie, de pilaren van artifici"ele intelligentie, het concept NoAI en verschillende demonstraties over deze onderwerpen.

\subsubsection{Eurofins - Testing met Robot Framework}
% https://pxl-digital.pxl.be/page/seminarie-eurofins-27-03 (down, zie BB)

\begin{tabular}{l l}
  Locatie: & PXL, Student Hub\\
  Datum: & 27/03/2019\\
  Duur: & 3 uur
\end{tabular}

Sepp Van Cauwenbergh, technical coach bij Eurofins, gaf een zeer praktisch seminarie over het schrijven van automatische testen. Aangezien dit in kleinere groep doorging, was de mogelijkheid er om veel oefeningen te maken en deze klassikaal te overlopen, waardoor we beter tot de kern van het onderwerp konden komen.

\subsubsection{Yappa - Piño: onder de schil, workshop introductie chatbot}
% https://pxl-digital.pxl.be/page/seminarie-yappa-07-04

\begin{tabular}{l l}
  Locatie: & Yappa, Eikaart 6 Bilzen\\
  Datum: & 03/04/2019\\
  Duur: & 3 uur
\end{tabular}

Het seminarie vond plaats in het bedrijf van Yappa, gelegen te Bilzen. Hier gaf Wesley Lancel, hub lead development bij Yappa, uitleg over Piño, de intern gebruikte chat assistant voor Slack. Met behulp van een demo werd verteld hoe Piño NLP gebruikt en hoe we op die manier zelf een chatbot konden maken.

\subsubsection{Quality@Speed - Performance testing: een levensnoodzakelijk iets}
% https://pxl-digital.pxl.be/page/seminarie-qualityatspeed-24-04

\begin{tabular}{l l}
  Locatie: & PXL, B241\\
  Datum: & 24/04/2019\\
  Duur: & 4 uur
\end{tabular}

Er werd zowel een introductie tot performance als een introductie tot een performance\hyp{}tool gegeven door Lisa\hyp{}Marie Van Bel, functional analyst/consultant bij Quality@Speed, en Gil Vanderhoven, ICT\hyp{}consultant bij Quality@Speed. Na deze introducties volgde een workshop om de geziene informatie over bijvoorbeeld Jmeter, praktisch te benaderen.

\subsubsection{Ordina - IT Security \& Data Privacy awareness session}
% https://pxl-digital.pxl.be/page/seminarie-ordina-15-05

\begin{tabular}{l l}
  Locatie: & PXL, B151\\
  Datum: & 15/05/2019\\
  Duur: & 2 uur
\end{tabular}

Tom Degol en Jelle Dauwe, security consultants bij Ordina, gaven een theoretisch seminarie over information security awareness en GDPR. De aangehaalde onderwerpen waren zowel op professioneel vlak als in het privéleven nuttig. Om iedereen aandachtig te houden, werd af en toe een quiz gehouden over de geziene onderwerpen van de presentatie.

\subsubsection{VRT NWS - Facebook en ik}
% https://www.vrt.be/vrtnu/a-z/facebook-en-ik/

\begin{tabular}{l l}
  Locatie: & PXL, Congress\\
  Datum: & 22/05/2019\\
  Duur: & 2 uur
\end{tabular}

Tim Verheyden, documentairemaker bij VRT NWS, doceerde een ontbijtseminarie. Hier bracht hij zijn onderzoek naar Facebook en hun omgang met privacy, democratie, ons welzijn, enz. aan. Ook de geschiedenis van hun ontstaan, de huidige stand van zaken en hun toekomstplannen lichtte hij ons toe.

\subsubsection{Infofarm - Internet of Things}
% https://pxl-digital.pxl.be/page/seminarie-infofarm-13-11

\begin{tabular}{l l}
  Locatie: & PXL, B334\\
  Datum: & 13/11/2019\\
  Duur: & 3 uur
\end{tabular}

Het seminarie van Inforfarm ging over IoT, een (momenteel) zeer populair onderwerp. De aangehaalde onderwerpen bestonden uit het visualiseren van sensor data, onregelmatigheidsdetectie en wijzigingsdetectie. De nadruk lag voornamelijk op het verschil in bruikbaarheid tussen verwerkte en onverwerkte data.

\subsubsection{Ericsson - Inleiding to 5G}
% https://pxl-digital.pxl.be/page/seminarie-ericsson-11-12

\begin{tabular}{l l}
  Locatie: & PXL, B323\\
  Datum: & 27/11/2019\\
  Duur: & 3 uur
\end{tabular}

Het seminarie dat door Tijs Van den Brande en Serge Vanhoffelen, customer project managers 5G POC bij Ericsson, gegeven werd, ging niet enkel over de werking van 5G, maar ook over de geschiedenis van mobiele data, met inbegrip van de overgang van 1G naar 5G en de werking van elk van deze. Er waren geen demo's of andere praktische elementen voorzien: dit seminarie bestond enkel uit een theoretisch deel.

\subsubsection{PwC - Secure development}
% https://pxl-digital.pxl.be/page/seminarie-pwc-04-12

\begin{tabular}{l l}
  Locatie: & PXL, B214\\
  Datum: & 04/12/2019\\
  Duur: & 3 uur
\end{tabular}

Het seminarie over secure development werd opgesplitst in twee delen: een theoretisch en een praktisch deel. Initieel werd er uitleg gegeven over of er rekening gehouden kan worden met het ontwikkelen van veilige software in korte cycli, ook wel sprints genaamd. In het tweede deel konden we deze technieken toepassen in een virtuele machine dat door PwC voorbereid was.