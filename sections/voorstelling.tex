% In dit onderdeel stel je jezelf voor:

% \begin{outline}
%   \1 Wie ben jij? Wat zijn je interesses?
%   \1 Over welke competenties en talenten beschik je (wat weet je, wat kan je goed, wat doe je graag)? Om een antwoord te geven op deze vragen kan je gebruik maken van je Thalento-rapport, je kernkwadrant en wat je geleerd hebt in de POP-sessies.
%   \1 Welke ambities heb je? Wat wil je over 3 of 5 jaar bereikt hebben?
%   \1 Wat kan ik en wat moet ik zeker nog kunnen?
% \end{outline}

Ik ben Vic Segers, een derdejaars student Toegepaste Informatica aan Hogeschool PXL met applicatie\hyp{}ontwikkeling als afstudeerrichting en AI \& Robotics als keuzetraject. Buiten school, speel ik ook basketbal in een ploeg en programmeer ik voor mijn persoonlijke projecten.

Momenteel speel ik 12 jaar basketbal bij de Bilzerse BC. Dit heeft mij niet enkel geholpen met fysiek maar ook mentaal gezond te zijn. Zo leerde ik samenwerken met anderen, mijzelf uiten en rekening houden met de mensen rondom mij.

Mijn huidige interesses zijn nu vooral rond AI en robotica, dit komt voornamelijk door het keuzetraject dat ik volg. Maar voor praktische uitwerkingen verkies ik robotica boven bijvoorbeeld machine learning, daarom dat ik mijn IT\hyp{}project en stage hierrond heb gedaan. In deze projecten werd een drone aangestuurd via zelfgeschreven code.

Van nature ben ik ook zeer nieuwsgierig, dat is ook te merken in de nieuwe technologie\"en die ik onder de knie probeer te krijgen. Zo ben ik nu \LaTeX{} aan het leren, hier is dit portfolio mee geschreven. Ik ben altijd op zoek naar iets nieuws om zo mijn kennis te verruimen en dit later te kunnen gebruiken wanneer nodig zou zijn.

Als ik een probleem moet oplossen zal ik eerst onderzoek doen naar hoe ik het dat het beste aanpak om mij er dan volledig op te storten. Ik sta wel altijd nog open voor de kennis van anderen en dit te verwerken in de uitwerking van het project. Ook kan ik mij zeer goed vastbijten op projecten die mij interesseren waardoor ik ook in mij vrije tijd er ook aan werk om tot een zo goed mogelijk resultaat te bekomen.

Volgend jaar zal ik schakelen naar een master in de toegepaste informatica. Eerst was ik aan het twijfelen of ik dit wel zou doen omdat het zeker niet vereist is om aan een job te komen, maar nu wil ik het zeker proberen. Vooral om dat ik de kans krijg om het te doen, ik met een modeltraject doorheen mijn bachelor opleiding ben gekomen en om later geen spijt te hebben dat ik het niet geprobeerd zou hebben. En natuurlijk voor de ervaring en mijn kennis rond AI en robotica te vergroten.

Hiervoor zal ik zeker nog veel moeten herhalen en bijleren voor wiskunde omdat dit toch al van het middelbaar geleden is dat ik hier nog intensief mee bezig ben geweest. Ik zal ook opnieuw veel theoretische leerstof moeten kunnen verwerken, wat in de professionele opleiding die ik nu volg minder het geval is.