% In dit onderdeel bespreek je een doordachte selectie van uitgevoerde activiteiten. Maak hierbij gebruik van de STARRT-methode.

% \begin{outline}
%   \1 Omschrijving
%     \2 Doelstelling(en) van de activiteit
%     \2 Eigen doelstelling(en)
%     \2 Optioneel teamsamenstelling (taakverdeling)
%   \1 Kern (showcase): min 500 woorden
%     \2 Verslag van de activiteit
%     \2 + screenshots, foto's, beeldmateriaal, \dots
%   \1 Reflectie opdracht: min 250 woorden
%     \2 Wat heb je gedaan?
%     \2 Heb je problemen ondervonden? Hoe heb je die aangemakt?
%     \2 Wat vond je ervan, hoe heb je het ervaren?
%     \2 Wat heb je geleerd?
%     \2 Wat waren jouw sterkte/zwakke punten?
%     \2 Wat zijn mogelijke linken met de opleiding?
%     \2 Waarom heb je deze opdracht geselecteerd voor de bespreking in jouw portfolio
% \end{outline}

% http://www.profi-leren.nl/files/oa_dc51.pdf

\subsection{Care\hyp{}athon - Ambulance Wens}
% https://pxl-digital.pxl.be/page/care-athon-2020#

% TODO: screenshots

De tweedaagse hackathon werd gehouden op de Corda Campus in gebouw 3 bij Cegeka. We werden hier ontvangen door Tristan Fransen en Francis Vos, hij gaf ons een korte introductie. Daarna maakte we kennis met de twee vertegenwoordigers van Ambulance Wens en Arno Barzan, de PXL\hyp{}begeleider. Arno Barzan gaf ons meer specifieke uitleg over hoe en wat er precies gerealiseerd moest worden. Mijn willekeurig ingedeelde team bestond uit vier studenten van applicatie\hyp{}ontwikkeling en één systeem\hyp{} en netwerkbeheerder.

Ik heb gekozen voor de uitdaging van Ambulance Wens, dit is een vzw die zich inzet voor mensen met een ongeneeslijke ziekte, die niet mobiel zijn en voelen dat het afscheid nabij komt. Ze zorgen dat deze mensen hun laatste wens nog in vervulling kunnen brengen met de medische ondersteuning die ze nodig hebben. Zo kunnen ze nog eens met volle teugen genieten van het leven en zich haast opnieuw goed voelen in hun vel.

Omdat de interne organisatie en het tentoonstellen van wensen nog ouderwets en stroef verliep, kregen wij de opdracht om een mobiele applicatie te ontwerpen voor de vzw. Zodat hierop niet alleen de pati\"enten een wens konden kiezen, maar dat er achterliggend ook verplegers en vrijwilligers konden deelnemen aan deze wensen. De nadruk lag dus vooral op de gebruiksvriendelijkheid en de consistentie van de applicatie.

Toen de hackathon daadwerkelijk van start ging, zijn we begonnen met iedereen in ons team een taak te geven. Zo moesten één student zich vooral focussen op het maken van de mock\hyp{}ups, twee studenten onderzoek doen naar realiseerbaarheid en restricties van buitenaf en de overige twee begonnen de applicatie te ontwikkelen. Zo kreeg ik de taak om de applicatie te ontwikkelen, maar ik hielp ook met beslissingen te nemen in verband met de mock\hyp{}ups. De applicatie werd geschreven voor Android in Android Studio met Java als programmeertaal. De mock\hyp{}ups werden uitgewerkt in de webapplicatie van Marvel.

Tegen het einde van de eerste dag zijn we als team tot besluit gekomen dat we geen volledige uitgewerkte applicatie konden presenteren in de gegeven tijd. Daarom zijn we de tweede dag ons meer gaan focussen op het afwerken en verfijnen van de mock\hyp{}ups, omdat we het belangrijker vonden om ons idee naar voren te brengen dan een half afgewerkt product. Dus vanaf toen zijn we met drie studenten gaan werken aan de mock\hyp{}ups en de andere twee studenten verder onderzoek naar elementen als server hosting en GDPR\hyp{}wetgeving.

\begin{figure}[!h]
  \centering
  \begin{subfigure}[h]{0.3\textwidth}
    \centering
    \includegraphics[width=0.76\textwidth]{care-athon/login.png}
  \end{subfigure}
  \begin{subfigure}[h]{0.3\textwidth}
    \centering
    \includegraphics[width=0.76\textwidth]{care-athon/registreer.png}
  \end{subfigure}
  \begin{subfigure}[h]{0.3\textwidth}
    \centering
    \includegraphics[width=0.76\textwidth]{care-athon/profiel.png}
  \end{subfigure}
\end{figure}

We konden regelmatig feedback gaan vragen aan Arno Barzan en aan de twee vertegenwoordigers van Ambulance Wens, dit gaf ons meer inspiratie en duidelijkheid om de applicatie zo goed mogelijk te ontwerpen aan hun vereisten. Zo wisten we welke informatie van pati\"enten, vrijwilligers en verplegers ze nodig hadden in hun applicatie. Maar ook hoe we het ontwerp in hun stijl konden verwerken.

Uiteindelijk moetsten we ook een presentatie maken die aan het eind van de hackathon gepresenteerd ging worden voor alle studenten van alle uitdagingen. Onze presentatie was een demo van de applicatie aan de hand van de mock\hyp{}ups. Maar omdat we ook ons onderzoek aan Ambulance Wens wouden geven hadden we een document gemaakt met alle informatie die we hadden gevonden en onze mening over bepaalde beslissingen die ze moesten maken.

Aan het einde van de hackathon kwamen iedereen die hieraan deelnam samen in een aula om te luisteren naar verschillende sprekers. Daarna waren de presentaties van alle teams, na dezen presentaties werden de winnaars van elke uitdaging bekend gemaakt en zij ontvingen een prijs afhankelijk van hun gekozen uitdaging.

Ik heb veel praktische zaken bijgeleerd van deze hackathon, zoals het maken van mock\hyp{}ups en het ontwikkelen van een applicatie met zeer specifieke restricties. Voordien had ik nog nooit zo een harde nadruk gelegd op het maken van mock\hyp{}ups, dit leerde ik van de student applicatie\hyp{}ontwikkeling met full\hyp{}stack development als keuzetraject. Maar ook het nadenken over hoe de applicatie er uit moet zien als er zoveel eisen en restricties bij komen kijken heeft mij anders doen denken over bepaalde ontwerp zaken en dus uiteindelijk ook over de ontwikkeling zelf.

Maar ook het deelnemen van een hackathon was nieuw voor mij, dat was een zeer verruimende ervaring om op een zeer korte tijd met een nieuw team een applicatie uit te gaan werken. Het is zeker iets wat in de opleiding moet blijven omdat ik naar mijn mening zeer veel zaken heb bijgeleerd op deze korte tijd. En het heeft mij doen inzien wat een software\hyp{}manager kan betekenen voor een team, omdat wij er geen hadden en ik dit dan kon vergelijken met teams dat er één of meerdere hadden. Zij hadden veel meer structuur en gingen meer georganiseerd te werk in vergelijking met ons team.

Dus het bekijken van de andere teams tijdens hun presentatie was ook leerzaam en tof om te zien hoe andere mensen nadenken over bepaalde zaken en zo mijn visie te verruimen. En omdat het voor een goed doel was en niet voor een bedrijf dat alleen maar meer winst wilt maken, zette ik mij meer in omdat ik zelf ook achter hun standpunten sta.

\begin{figure}[!h]
  \centering
  \begin{subfigure}[h]{0.3\textwidth}
    \centering
    \includegraphics[width=0.76\textwidth]{care-athon/wensen.png}
  \end{subfigure}
  \begin{subfigure}[h]{0.3\textwidth}
    \centering
    \includegraphics[width=0.76\textwidth]{care-athon/wens.png}
  \end{subfigure}
  \begin{subfigure}[h]{0.3\textwidth}
    \centering
    \includegraphics[width=0.76\textwidth]{care-athon/notificatie.png}
  \end{subfigure}
\end{figure}

\subsection{Studiereis: Berlijn (Technische Universit\"at Berlin)}
% htts://pxl-digital.pxl.be/page/studentenreis-2020-berlijn2

Als internationaliseringsonderdeel was er de mogelijkheid om op studiereis te gaan met medestudenten. Ik heb deze kans gegrepen en heb gekozen voor de studiereis naar Berlijn. Er waren verschillende reizen, één naar Amsterdam en twee naar Berlijn. Die naar Amsterdam heb ik niet gekozen omdat ik al vaker naar die stad ben gegaan en Cisco, wat mij niet echt aanspreekt, was daar de hoofdzaak. Dus er bleven nog twee reizen naar Berlijn over, en ik verkoos het om naar de universiteit te gaan dan deel te nemen aan de makathon.

De reis begon natuurlijk met een lange busrit, hier leerde we elkaar al wat beter kennen en wisten we wie er allemaal mee ging. Eenmaal aangekomen konden we inchecken in de kamer van het hotel en hadden we de rest van de avond nog vrij om Berlijn te gaan verkennen of om uit te rusten voor de drukke dagen die we tegemoet stonden.

De eerste echte dag in Berlijn werden we opgesplitst in twee groepen om afwisselend de geplande activiteiten uit te voeren. Zo vertrok de ene groep naar de Stasi\hyp{}gevangenis Hohensch\"onhausen en de andere naar de Tempelhof luchthaven. Na de middag wisselde de groepen van activiteit. In de gevangenis kregen we een rondleiding van een ex\hyp{}gevangene, dit was een voor\hyp{} en een nadeel. Zijn Engels was niet zo goed waardoor hij in het begin moeilijk te verstaan was, maar nadien kon hij zeer mooi zijn ervaringen en gevoelens delen met ons wat het zeer interessant maakte. Hij had een zeer fascinerende houding tegenover de gevangenis en de bewakers van toen.

Daarna kregen we een rondleiding in de luchthaven, wat vroeger een van de grootste gebouwen van Europa was. Hier hebben we natuurlijk veel moeten wandelen om het volledige gebouw te zien, maar dat was zeker de moeite. De luchthaven toen wij het bezochten was volledig leeg en verlaten op een paar ruimtes na. Dankzij de goede uitleg en aanvullende attributen kregen we toch een goed idee over hoe de luchthaven functioneerde in zijn hoogtepunt en waarom dat hij nog zo goed intact is gebleven geduurde de oorlog. En ook wat de zijn huidige functie is en wat er in de toekomst nog mee zou kunnen gebeuren.

De volgende dag was er een toeristische rondleiding gepland in Berlijn. Zo reden we eerst met de bus door Berlijn en gaf een gids uitleg over monumenten en straten waar we langs of door reden. Na deze busrit gingen we te voet verder met de gids. Zo bezochten we monumenten als de Berlijnse Muur, Checkpoint Charlie, Potsdamer Platz, de Berliner Dom, het Holocause monument en nog veel meer. De gids gaf een goede uitleg over het verleden van deze monumenten maar ook de huidige betekenis en waarde ervan. Normaal gezien ben ik geen fan van een rondleiding met een gids, maar hier was ik zeer aangenaam van verrast omdat we ook veel hebben kunnen zien op een korte tijd met een nuttige uitleg.

\begin{figure}[!h]
  \centering
  \begin{subfigure}[h]{0.48\textwidth}
    \includegraphics[width=\linewidth]{berlijn/tempelhof.jpg}
    \caption{Tempelhof luchthaven}
  \end{subfigure}
  \begin{subfigure}[h]{0.48\textwidth}
    \includegraphics[width=\linewidth]{berlijn/holocaust_monument.jpg}
    \caption{Holocaust monument}
  \end{subfigure}
\end{figure}

De laatste dag in Berlijn zouden we normaal naar de Technische Universit\"at Berlin gaan en daar een seminarie volgen rond Future Security Lab, maar dit is niet doorgegaan om een voor ons onbekende reden. Na de middag kregen we daar ook een seminarie dat nog niet bekend was, maar deze hebben we uiteraard ook niet kunnen volgen. Omdat onze begeleiders deze late annulering niet hadden voorzien, mochten we deze dag zelf inplannen. Ik ben dan met een paar vrienden Berlijn nog wat verder gaan verkennen en op het gemak wat genieten en uitrusten van de vermoeide afgelopen dagen. Na een korte nacht konden vlug uitchecken van het hotel en ontbijten voor we de bus weer opstapte richting Hasselt.

De studiereis was zeker geslaagd voor mij, alleen vond ik het jammer dat we niet naar de universiteit zijn kunnen gaan. Ik heb mijn medestudenten zeker beter leren kennen en daar nieuwe vrienden gemaakt, wat ik niet had verwacht voor ik vertrok op de reis. Ik ga wel vaker op reis, dus ik wist wel waar ik mij aan moest verwachten, maar om dit met een school te doen is wel nog een andere ervaring.

Op de onvoorziene omstandigheden na, zou ik de reis zeker opnieuw doen en aanraden voor andere studenten. Het enige waar ik niet naar zou uitkijken zijn de busritten, maar die horen er natuurlijk wel bij. Het was zeker wel een vermoeiende reis, korte nachten en lange dagen, zo leer je elkaar wel veel beter kennen op een korte periode.

Ik heb deze opdracht opgenomen in mijn portfolio omdat deze studiereis toch wel een grote indruk heeft nagelaten en het een groot deel uitmaakte van mijn derde en belangrijkste jaar op Hogeschool PXL. Ook omdat op gebied van mijn medestudenten leren kennen, dit toch een van de meest impactvolle elementen is en het plezier maken met elkaar.

\begin{figure}[!h]
  \centering
  \begin{subfigure}[h]{0.48\textwidth}
    \includegraphics[width=\linewidth]{berlijn/berliner_dom.jpg}
    \caption{Berliner Dom}
  \end{subfigure}
  \begin{subfigure}[h]{0.48\textwidth}
    \includegraphics[width=\linewidth]{berlijn/potsdamer_platz.jpg}
    \caption{Portzdamer Platz}
  \end{subfigure}
  \begin{subfigure}[h]{0.48\textwidth}
    \includegraphics[width=\linewidth]{berlijn/uitzicht_reichstag.jpg}
    \caption{Uitzicht vanaf Reichstag}
  \end{subfigure}
  \begin{subfigure}[h]{0.48\textwidth}
    \includegraphics[width=\linewidth]{berlijn/brandenburger_tor.jpg}
    \caption{Brandenburger Tor}
  \end{subfigure}
\end{figure}

\subsection{AariXa - Docker for Dev and Ops}
% https://pxl-digital.pxl.be/page/seminarie-aarixa-27-02

Het eerste seminaire dat ik volgde voor iTalent ging over Docker, een tool om applicatie in containers te runnen. De spreker Sven Luts, software engineer bij AariXa, gaf uitleg over wat deze tool is, waarom het gebruikt wordt en hoe het achterliggend werkt. Hij maakte ook gebruik van demo's om de aangehaalde onderwerpen te verduidelijken.

In het begin stelde de spreker zich voor. Hij deelde zijn werkplek, zijn hobby's en zijn specialiteiten op gebied van informatica met ons. De introductie van Docker volgde zijn introductie. Docker is een tool gemaakt om het aanmaken, het deployen en het runnen van applicaties in containers te versimpelen.

Vervolgens gaf de spreker een uitleg over waarom je Docker moet gebruiken. Het vereenvoudigt niet alleen het maken en het deployen van software, maar het is ook veel sneller dan een virtuele machine. Het voorkomt de bekende uitspraak: "It works on my machine", dit bekomt Docker door in een afgeschermde en ge"isoleerde omgeving te werken. Deze omgeving bevatten alle benodigde dependencies en requirements. Docker bevat ook een ingebouwd version tracking door gebruik te maken van Docker tags. Een applicatie in een Docker container zal op eender wel systeem exact hetzelfde runnen ongeacht van de reeds ge"installeerde software op het host systeem.

Het volgende onderwerp was een Docker container, wat ze zijn en hoe je er aan kan komen. Een Docker container kan vergeleken worden met een virtuele machine, in de zin dat het ge"isoleerd is van het host systeem en het alle benodigdheden die de applicatie nodig heeft om te runnen. Het verschil met een virtuele machine is dat een Docker container geen operating system moet gaan virtualiseren, wat een virtuele machine wel moet doen. Een Docker container gebruikt het operating system van het host systeem. Hierdoor is een Docker container aanmerkelijk kleiner is formaat dan een virtuele machine en heeft het praktisch geen opstarttijd.

Om een Docker container te bekomen is er eest een image nodig. Aan een Docker image kan je op twee manieren komen: je downloadt het van het internet of je maakt er zelf een. Van een Docker registry kan een Docker image gepulld worden, dit is het equivalent van downloaden van het internet. De image kan ook zelf gebuild worden door het builden van een Dockerfile, een bestand waarin het buildproces van een image gedefinieerd wordt. Een Docker container verkrijg je als je deze Docker image runt. In een Docker container kunnen commits met veranderingen naar de image worden gedaan, waarna je deze Docker image weer naar een registry kan pushen.

De spreker toonde na deze uitleg hoe we Docker zelf konden installeren op ons systeem. Docker werkt achterliggend met Linux en vinden er geen problemen plaats bij een installatie op een Linux systeem. Als de installatie plaats vindt op een Mac of een Windows systeem, gaat er in de achtergrond een virtuele machine met Linux opgestart worden die Docker beheert. Hierop volgde de uitleg van de basis commando's om Docker te beheren en monitoren.

\begin{figure}[!h]
  \centering
  \includegraphics[width=0.72\linewidth]{images/docker/docker_vs_vm.png}
\end{figure}

Het seminarie sloot af met een korte demo over hoe we snel en gemakkelijk Docker commando's konden uittesten in een webapplicatie. Hiervoor moest er dus niets ge"installeerd worden. Deze demo functioneerde als de "Hello World!"\ van Docker. Na de demo gaf de spreker nog een kort vooruitzicht naar wat er in de toekomst met Docker kon gebeuren en gerealiseerd worden.

Het onderwerp van het gegeven seminarie vond ik boeiend, maar ik kon niet plaatsen hoe ik deze technologie zou kunnen gebruiken binnen mijn toekomstige projecten. Echter, toen ik gedurende de vakken rond artifici"ele intelligentie en robotica in aanraking kwam met Docker, zag ik het nut van deze Docker in. Dit gaf mij niet alleen een grote voorsprong op mijn medestudenten die hier nog niet mee in aanraking gekomen waren, maar het deed mij ook beseffen dat het een zeer praktische technologie is waar ik in de toekomst ook nog gebruik van zou maken.

Ondertussen zijn mijn grootste projecten in Docker gemaakt. Namelijk mijn IT\hyp{}project en mijn stage hadden Docker als basis. Het seminarie en de technologie hebben mij ge"inspireerd om ge"isoleerde applicaties te gaan schrijven. Ook ga ik heel anders om met dependencies dan ik voordien deed, ik zorg dat ik altijd een stabiele versie van een bepaalde software gebruik en dat er geen conflicten zijn tussen de gebruikte software. Docker geeft in een bepaalde zin meer vrijheid om te experimenteren met het project. Doordat je altijd terug kan gaan naar een stabiele versie waarin de applicatie werkend is.

De spreker bracht deze presentatie zeer rustig en haalde goede punten aan. Door niet af te dwalen maar toch wat leuke weetjes in zijn presentatie te stoppen, verloor hij niemand zijn aandacht. Hij deelde zijn eigen ervaring met Docker. Zijn projecten konden zonder probleem op elke computer met Docker runnen, zonder dat hij hier enige extra moeiten in had gestoken.

Naar mijn mening is Docker zo handig dat ik het in een laat eerste\hyp{} of vroeg tweedejaarsvak aan zou halen. Dit kan makkelijk in vakken zoals Desktop OS of Server OS Essentials, waar we al kennis maakten met Linux. De tijd die ik al gewonnen heb met Docker en de vrijheid die je er van krijgt, zorgen ervoor dat Docker momenteel een van mijn favoriete en meest gebruikte technologie is.

Spreker: Sven Luts, software engineer bij AariXa, \url{https://be.linkedin.com/in/svenluts/nl}


\begin{figure}[!h]
  \centering
  \includegraphics[width=0.74\linewidth]{images/docker/flow.png}
\end{figure}

\subsection{\LaTeX - Softwaresysteem voor het zetten van documenten}
\LaTeX{} is een softwaresysteem dat je kan vergelijken met een tekstverwerker zoals Microsoft Word. Het is zeer populair in de wetenschappelijke wereld, want het blinkt uit in het maken van technische documenten. Het idee van \LaTeX{} is WYMIWYG, "What you mean is what you get", en dus een markup\hyp{}taal.

Ik kwam als eerst in contact met \LaTeX{} door de tekstverwerker van MacOS genaamd Pages. Hier kun je wiskundige vergelijkingen invoegen via \LaTeX{} en MathML. Eerst kopi\"eerde ik formules die ik online vond of ik converteerde ze via een online converter. Daarna zag ik dat de taal niet zo moeilijk te begrijpen was en zo kon ik mijn eigen formules uitschrijven.

Later kwam ik er achter dat \LaTeX{} niet enkel gemaakt is voor wiskundige formules weer te geven, maar ook kan gebruikt worden als een volwaardige tekstverwerker. Omdat het zeer veel gebruikt wordt om papers mee te schrijven, had ik besloten om mijn eindwerk hiermee te schrijven.


