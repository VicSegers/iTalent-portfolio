\subsubsection{Sinterklaas programming challenge}

\begin{tabular}{l l}
  Locatie: & n.v.t.\\
  Datum: & 03/12/2018 - 04/12/2018\\
  Duur: & 2 uur
\end{tabular}

Tijdens het vak Java Advanced, kregen we de mogelijkheid om aan een programming challenge deel te nemen. Er moest aan de hand van een gegeven technologie een probleem opgelost worden voor de sint. Ik heb niet enkel deelgenomen aan deze challenge, maar ik heb hiervoor ook een prijs gewonnen.

\subsubsection{\LaTeX{} - Softwaresysteem voor het zetten van documenten}

\begin{tabular}{l l}
  Locatie: & n.v.t.\\
  Datum: & 24/02/2020 - ??/04/2020\\
  Duur: & 20 uur
\end{tabular}

Na veel te lezen over \LaTeX{} kreeg ik de kriebels om het zelf te leren en er mee te werken. Dit softwaresysteem gebruikte ik dan om niet enkel mijn eindwerk, maar ook mijn portfolio voor iTalent te schrijven. Zo legde ik mijn eigen op om deze technologie zeer goed onder de knie te moeten krijgen om het te gebruiken voor zulke belangrijke documenten.

\subsubsection{Website portfolio iTalent}

\begin{tabular}{l l}
  Locatie: & n.v.t.\\
  Datum: & 12/04/2020 - ??/04/2020\\
  Duur: & 12 uur
\end{tabular}

Voor het vak iTalent in mijn bachelor opleiding moest er een portfolio gemaakt worden. Om deze online beschikbaar te hebben en mooier te presenteren dan een tekstdocument heb ik er een website van gemaakt. Zo leerde ik om een statische responsive website te maken met Bootstrap en het te hosten met Github Pages. Deze technologie\"en had ik voordien nog niet gebruikt.