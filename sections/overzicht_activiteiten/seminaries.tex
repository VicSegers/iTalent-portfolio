\subsubsection{AariXa - Docker for Dev and Ops}
% https://pxl-digital.pxl.be/page/seminarie-aarixa-27-02

\begin{tabular}{l l}
  Locatie: & PXL, B141\\
  Datum: & 27/02/2019\\
  Duur: & 3 uur
\end{tabular}

Het seminarie werd gegeven door Sven Luts, een software engineer bij AariXa. De spreker gaf een introductie over Docker en hoe het achterliggend werkt. En door middel van demo's werden de nodige commando's duidelijk gemaakt aan ons.

\subsubsection{Belfius - Introduction to Artificial Intelligence Through Practice}
% https://pxl-digital.pxl.be/page/seminarie-belfius-20-03

\begin{tabular}{l l}
  Locatie: & PXL, B151\\
  Datum: & 20/03/2019\\
  Duur: & 3 uur
\end{tabular}

We kregen een introductie tot artificiële intelligentie door Jerome Fortias, AI practice leader bij BrightKnight. Er kwamen verschillende onderwerpen aan bod zoals de geschiedenis en de pilaren van AI, wat NoAI is en verschillende demonstraties hierover.

\subsubsection{Eurofins - Testing met Robot Framework}
% https://pxl-digital.pxl.be/page/seminarie-eurofins-27-03 (down, zie BB)

\begin{tabular}{l l}
  Locatie: & PXL, Student Hub\\
  Datum: & 27/03/2019\\
  Duur: & 3 uur
\end{tabular}

Sepp Van Cauwenbergh, technical coach bij Eurofins, gaf een zeer praktisch seminarie over automatisch testen schrijven. Omdat er niet veel plaatsen waren, was de mogelijkheid er om veel oefeningen te maken en klassikaal te overlopen.

\subsubsection{Yappa - Piño: onder de schil, workshop introductie chatbot}
% https://pxl-digital.pxl.be/page/seminarie-yappa-07-04

\begin{tabular}{l l}
  Locatie: & Yappa, Eikaart 6 Bilzen\\
  Datum: & 03/04/2019\\
  Duur: & 3 uur
\end{tabular}

Het seminarie vond plaats in het bedrijf van Yappa, gelegen te Bilzen. Hier gaf Wesley Lancel, hub lead development bij Yappa, uitleg over Piño, de intern gebruikte chat assistant voor Slack. Samen met behulp van een demo werd er verteld hoe Piño NLP gebruikt en hoe we zo een chatbot zelf konden maken.

\subsubsection{Quality@Speed - Performance testing: een levensnoodzakelijk iets}
% https://pxl-digital.pxl.be/page/seminarie-qualityatspeed-24-04

\begin{tabular}{l l}
  Locatie: & PXL, B241\\
  Datum: & 24/04/2019\\
  Duur: & 4 uur
\end{tabular}

Er werd zowel een introductie tot performance als een introductie tot een performance\hyp{}tool gegeven door Lisa\hyp{}Marie Van Bel, functional analyst/consultant bij Quality@Speed, en Gil Vanderhoven, ICT\hyp{}consultat bij Quality@Speed. Na deze introducties was er een workshop om de geziene informatie over bijvoorbeeld Jmeter, praktisch te benaderen.

\subsubsection{Ordina - IT Security \& Data Privacey awareness session}
% https://pxl-digital.pxl.be/page/seminarie-ordina-15-05

\begin{tabular}{l l}
  Locatie: & PXL, B151\\
  Datum: & 15/05/2019\\
  Duur: & 2 uur
\end{tabular}

Tom Degol en Jelle Dauwe, security consultatns bij Ordina, gaven een theoretisch seminarie over information security awareness en GDPR. De aangehaalde onderwerpen waren zowel op professioneel vlak als in het privéleven nuttig. En om iedereen bij de les te houden hielden ze af en toe een quiz over de gegeven informatie.

\subsubsection{VRT NWS - Facebook en ik}
% https://www.vrt.be/vrtnu/a-z/facebook-en-ik/

\begin{tabular}{l l}
  Locatie: & PXL, Congress\\
  Datum: & 22/05/2019\\
  Duur: & 2 uur
\end{tabular}

Tim Verheyden, documentairemaker bij VRT NWS, gaf aan ons een ontbijtseminarie. Hier bracht hij zijn onderzoek naar Facebook en hun omgang met privacy, democratie, ons welzijn, enzovoort. Ook de ontstaangeschiedenis, de huidige stand van zaken en hun toekomst(plannen) lichtte hij ons toe.

\subsubsection{Infofarm - Internet of Things}
% https://pxl-digital.pxl.be/page/seminarie-infofarm-13-11

\begin{tabular}{l l}
  Locatie: & PXL, B334\\
  Datum: & 13/11/2019\\
  Duur: & 3 uur
\end{tabular}

Het seminarie van Infofarm ging over IoT, een zeer populair onderwerp tegenwoordig. De verschillende aangehaalde onderdelen waren het visualiseren van sensor data, anomaliedetectie en wijzigingsdetectie. De nadruk lag vooral op het verschil in bruikbaarheid tussen verwerkte en onverwerkte data.

\subsubsection{Ericsson - Inleiding to 5G}
% https://pxl-digital.pxl.be/page/seminarie-ericsson-11-12

\begin{tabular}{l l}
  Locatie: & PXL, B323\\
  Datum: & 27/11/2019\\
  Duur: & 3 uur
\end{tabular}

Het door Tijs Van den brande en Serge Vanhoffelen, customer project managers 5G POC bij Ericsson, gegeven seminarie ging niet enkel over de werking van 5G, maar ook over de geschiedenis ervan, dus hoe het van 1G naar 5G is gegaan en de werking van elke tussenstap. Er waren geen demo's of andere praktische elementen voorzien, enkel de theorie werd gegeven.

\subsubsection{PwC - Secure development}
% https://pxl-digital.pxl.be/page/seminarie-pwc-04-12

\begin{tabular}{l l}
  Locatie: & PXL, B214\\
  Datum: & 04/12/2019\\
  Duur: & 3 uur
\end{tabular}

Het seminarie over secure development was opgesplits in twee delen, een theoretisch en een praktisch. Er werd eerst uitleg gegeven over er rekening kan gehouden worden met het ontwikkelen van veilige software in korte cycli oftewel sprints. In het tweede deel konden we deze technieken toepassen in een VM dat door PwC was voorbereid.