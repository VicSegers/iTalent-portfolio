% In dit onderdeel stel je jezelf voor:

% \begin{outline}
%   \1 Wie ben jij? Wat zijn je interesses?
%   \1 Over welke competenties en talenten beschik je (wat weet je, wat kan je goed, wat doe je graag)? Om een antwoord te geven op deze vragen kan je gebruik maken van je Thalento-rapport, je kernkwadrant en wat je geleerd hebt in de POP-sessies.
%   \1 Welke ambities heb je? Wat wil je over 3 of 5 jaar bereikt hebben?
%   \1 Wat kan ik en wat moet ik zeker nog kunnen?
% \end{outline}

Ik ben Vic Segers, een derdejaars student Toegepaste Informatica aan Hogeschool PXL met applicatieontwikkeling als afstudeerrichting en AI \& Robotics als keuzetraject. Buiten school speel ik ook basketbal in een ploeg en programmeer ik voor mijn persoonlijke projecten.

Momenteel speel ik reeds 12 jaar basketbal bij de Bilzerse BC. Dit heeft mij niet enkel geholpen met mijn fysieke, maar ook mijn mentale gezondheid. Hieruit leerde ik samenwerken met anderen, mezelf uiten en rekening houden met mijn ploeggenoten.

Mijn huidige interesses spelen zich vooral af rond AI en robotica. Dit is voornamelijk ten gevolge van het keuzetraject dat ik volg. Voor praktische uitwerkingen verkies ik echter robotica boven bijvoorbeeld machine learning. Om diezelfde reden heb ik ervoor gekozen om gedurende mijn IT\hyp{}project en stage deze tak van de IT verder toe te passen en aan te leren. Gedurende deze projecten leerde ik om code te schrijven die ervoor zorgde dat een drone op onafhankelijke basis kan rondvliegen en zijn omgeving in kaart kan brengen.

Van nature ben ik ook een zeer nieuwsgierig persoon. Dit uit zich onder meer door de nieuwe technologie"en die ik onder de knie probeer te krijgen. Zo ben ik momenteel \LaTeX{} aan het leren, hetgeen ik onder andere gebruikt heb om mijn portfolio mee te schrijven. Ik ben ook steeds op zoek naar nieuwe informatie en technologie"en om mijn kennis te verruimen en dit later te kunnen toepassen wanneer dit nuttig wordt geacht. Ik probeer ook steeds de kennis van mijn medestudenten te raadplegen wanneer ik zelf geen oplossing kan vinden en kan deze kennis vervolgens ook toepassen in projecten.

Wanneer ik voor een probleem kom te staan, zal ik steeds eerst onderzoeken wat het beste plan van aanpak is, om me er vervolgens volledig in te storten. Wanneer een project me echter interesseert, kan ik me er zelfs dermate in vastbijten dat ik zelfs tijdens mijn vrije tijd er nog aan verder werk, om zo het best mogelijke resultaat te bekomen.

Volgend jaar zal ik schakelen naar een master in de toegepaste informatica. Initieel was ik aan het twijfelen of ik dit wel zou doen, omdat het zeker geen vereiste is om aan een job te geraken - aangezien er voor informatici meer dan voldoende jobs te vinden zijn -, maar nu wil ik het zeker een kans geven. De redenen waarom ik toch gekozen heb om mijn opleiding verder te zetten zijn ten eerste - en vooral - omdat ik er de kans toe krijg. Ten tweede heb ik mijn bachelor in modeltraject kunnen afwerken, dus verwacht ik ook geen uitgesproken problemen met mijn verdere studies. Ten slotte wil ik het ook proberen om zeker te zijn dat ik later geen spijt zal hebben dat ik het niet geprobeerd zou hebben. Uiteraard zal deze opleiding mijn kennis en ervaringen rond AI en robotica uitbreiden. Ook zal deze opleiding een aantal uitdagingen met zich meedragen. Mijn wiskunde zal ten eerste terug herhaald moeten worden aangezien ik hier al van het middelbaar niet meer echt mee bezig ben geweest. Ook zal ik terug veel theoretische leerstof moeten verwerken, hetgeen in mijn huidige, professionele opleiding minder veel aan bod kwam.